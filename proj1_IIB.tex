\documentclass[12pt]{article}

\usepackage{a4wide}
\usepackage{amsmath}
\usepackage{amssymb}
\usepackage{artmacros}
\usepackage{bm}
\usepackage{graphicx}

\begin{document}
\title{Analysis of ECG Signals \\ Part B: Heart Rate Variability \\ Project in SSY130 }
\author{Tomas McKelvey}

\maketitle{}
\section{Introduction}
In part one of the project the analysis of the heart rate was made
based on a very simplified model, i.e.,\ the assumption that the heart
beats are periodic. However, a closer analysis of the heart beats
reveals that the beats are appearing at irregular time instances. This
variability has many physiological sources and can be of clinical use
for monitoring and diagnostic  purposes. The overall phenomenon of the
irregularity of the heart beats is called \emph{heart rate variability
  (HRV)} and can be analyses in several ways. In this project you will
develop some signal processing tools and use them to assess the
HRV found in some ECG recordings provided. You find all data and
scripts on the git page \texttt{https://git.chalmers.se/tomas.mckelvey/ssy130-ecg-analysis.git}

\subsection{ECG Data}
\label{sec:ecg-data}

The data for this project task are long term ECG readings obtained
with a sampling frequency of 128 Hz. The origin of the data is ``MIT-BIH Normal
Sinus Rhythm Database''  from the PhysioNet database
\cite{goldberger2000physiobank}.  Note that some of the recordings
have poor quality for the first minutes of recordings. You can
disregard these data in your analysis. 

\subsection{RR Interval}
\label{sec:r-r-interval}

All heart rate variability analysis depart from the \emph{RR
  interval} which is the time between two consecutive heart
beats. The electrical pattern in an ECG signal has several parts which
origins from the different phases a heart experience during a heart
beat. The different phases are named P-Q-R-S-T waves, see illustration
in Figure~\ref{fig:pqrst}. The R-wave is the segment which is mostly
pronounced and is, in a normal ECG, a fast large peak. Hence, to pin down an ECG
complex to a specific time instance the peak of the R-wave is often used and
the time elapsed between to consecutive R-peaks is hence called the RR
interval.

In order to successfully apply HRV analysis the RR intervals need to
be accurately estimated. This essentially boils down to the have a
robust method to detect the location of the peak of the R-wave.   

\subsection{Tasks}
\begin{enumerate}
\item The script \texttt{hrvRRdetect.m} illustrates how the ECG data
  is loaded and how results can be plotted. Complete the file with
  code which detects the R peaks and plot the R-peak detections in the
  ECG graph. The R-wave is characterised by rapid changes, i.e. the
  derivative is high. This property could be utilized in the detection
  bny using a FIR filter which calculates the derivative of the signal.  
  The following issues should be considered.
  \begin{itemize}
  \item For some individuals the ECG has a high gain and for some
    lower gain. The peak detection method should automatically
    compensate for this.
  \item In some cases there is a slowly varying bias, i.e.\ the
    baseline is drifting. Remove this effect by some appropriate filter.
  \end{itemize}
  Using a FIR local model approach with monomial basis functions and
  using the derivative as the output could be one possibility. 
\item The respiratory activity has an impact on the RR intervals and
  the IHR. Analyze this in some of the ECG traces and try to determine
  the respiratory rate.   
\item It is clear that it is almost impossible to find a filtering
  and peak detection method which is fail-safe. In order to further analyze
  the RR intervals the data need to be curated by removing so called
  outliers in the data. In the sequence of RR intervals should an
  outlier be removed or replaced by neighbour data?
\item We can define the instantaneous heart rate (IHR) as the rate (in
  beats/min) that a given RR interval indicates. Derive the
  expression of the IHR if the RR interval is $q$ samples and the
  sampling frequency is $f_s=128$ Hz. Use this calculation when
  scale the IHR when plotting it. 
\item For each new R-peak detected a new RR interval can be
  determined. This implies that the RR interval data are obtained at a
  non-equidistant time instances. Discuss how such data should be
  plotted and further analyzed with respect to this issue. 
\end{enumerate}

\subsection{Analysis of  long term HRV data}
The variability of the RR intervals around an average gives some
indications of the status of the individual. A high variability of the
RR interval is a recognized index of the ability of the cardiovascular
system to cope with environmental challenges.
\subsection{Tasks}
\begin{enumerate}
\item Create a script that segment the ECG data into 1 minute
  parts. For each segment derive the RR intervals and calculate the
  mean value and the standard deviation. This analysis will generate
  two new time series with a sampling interval of 1 minute. Plot the result and discuss
  the long term variation seen in the data.  
\end{enumerate}

\subsection{(Optional) Long term frequency analysis of the
  RR intervals}
\label{sec:optional-long-term}
The RR intervals time-series can also be analysed using Fourier
Analysis to reveal various periodic behavior. RR intervals manifest
short-term oscillations in a frequency range between 0 and 0.5 Hz,
which appear to be the result of
intrinsic autonomic rhythms and of respiratory inputs. Spectral
analysis of RR intervals provides an estimate on how power (i.e.,
variance) of the signal is distributed as a function of frequency. RR
intervals appear to be organized in three major components, the
high-frequency (HF) ($>$ 0.15 Hz) respiratory band, the low-frequency
(LF) band (around 0.1 Hz) and the very–low-frequency (VLF) band
(0.003-0.039 Hz). The HF components of RR variability primarily
reflect the respiration-driven modulation of sinus rhythm. Feel free
to experiment with your code and see if the data reveals the HF, LF
and VLF components mentioned above. 


\begin{figure}
  \centering
  \includegraphics[width=0.5\columnwidth]{SinusRhythmLabels.png}
  \caption{ECG wave and the sub-waves. Created by Agateller (Anthony Atkielski), converted to svg by atom. - SinusRhythmLabels.png, Public Domain, https://commons.wikimedia.org/w/index.php?curid=1560893 }  \label{fig:pqrst}
\end{figure}


\label{sec:introduction}
\bibliographystyle{IEEEtran}
\bibliography{/Users/mckelvey/Downloads/My_Library.bib}
\end{document}
